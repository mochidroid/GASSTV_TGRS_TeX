\section{Introduction}
\IEEEPARstart{H}{yperspectral} (HS) images, with their rich spectral information over 100 bands, are widely applied in diverse fields such as agriculture, mineralogy, astronomy, and biotechnology~\cite{Borengasser2007HSIApplications,Grahn2007Techniques, Thenkabail2016VegetationOverview,Lu2020AgricultureOverview}. Despite their potential, HS images are contaminated with noise during acquisition, which adversely affects the performance of subsequent analyses, including anomaly detection ~\cite{Matteoli2014Anomaly, Su2022Anomaly}, classification~\cite{Ghamisi2017Classification}, and unmixing~\cite{Bioucas-Dias2012UnmixingOverview, Ma2014UnmixingOverview}. Therefore, HS image denoising is an essential preprocessing step~\cite{Shen2015DenoisingOverview, Rasti2018DenoisingOverview, Shen2022DenoisingOverview, Naganuma2022Destriping}.


For HS image denoising tasks, Spatio-Spectral Total Variation (SSTV)~\cite{Aggarwal2016SSTV} is widely known as a powerful regularization method applied in state-of-the-art methods~\cite{Fan2018SSTV-LRTF, Wang2018LRTDTV, Ince2019GLSSTV, Takeyama2020HSSTV, Wang2021l0l1HTV, Takemoto2022GSSTV, Takemoto2023S3TTV, takemoto2024spatiospectral}. SSTV effectively captures the both spatial and spectral piecewise smoothness of HS images by leveraging the $\ell_{1}$-norm of spatial differences computed after spectral ones. However, SSTV evaluates neighborhood differences uniformly, which is limited to preserve 1) complex spatial structures and 2) spectral jumps.

%In this field of HS image denoising, Spatio-Spectral Total Variation~\cite{Aggarwal2016SSTV} is widely known as a powerful regularization method that captures spatial and spectral properties on HS images, and it is also applied in state-of-the-art methods~\cite{Fan2018SSTV-LRTF, Wang2018LRTDTV, Ince2019GLSSTV, Takeyama2020HSSTV, Wang2021l0l1HTV, Takemoto2022GSSTV, Takemoto2023S3TTV, takemoto2024spatiospectral}.
%SSTV characterizes the both spatial and spectral piecewise smoothness by defining the $\ell_{1}$-norm of the spatial differences after computing spectral ones.
%However, SSTV uniformly evaluates the neighborhood differences in the vertical, horizontal, and spectral directions, which limits its ability to preserve 1) complex spatial structures and 2) spectral jumps.


% To address the spatial limitations of SSTV, Graph Spatio-Spectral Total Variation (GSSTV)~\cite{Takemoto2022GSSTV} incorporates a spatial graph-based weighted difference operator into SSTV.
% In GSSTV, a method is established to construct a spatial graph that explicitly reflects the spatial structure of the target HS image from the noisy HS image.
% By utilizing the spatial graph, GSSTV successfully captures more complex spatial structures compared to SSTV.
% However, GSSTV leaves noise in regions where the spatial graph weights are close to zero, and the spectral limitations of SSTV remain unresolved.
% Therefore, we pose the research question: can we enhance the noise removal capabilities of SSTV while preserving complex spatial structures and spectral jumps?

As a promising extension of SSTV, Graph Spatio-Spectral Total Variation (GSSTV)~\cite{Takemoto2022GSSTV} integrates a spatial graph-based weighted difference operator into SSTV, addressing challenge 1) i.e., preserving complex spatial structures.
However, GSSTV leaves residual noise in low-weight regions of the spatial graph, and does not address the spectral limitations of SSTV. Now a natural question arises: \textit{can we design a regularization method that enhances the noise removal capabilities of SSTV while capturing complex spatial structures and spectral jumps?}
%However, GSSTV leaves residual noise in low-weight regions of the spatial graph, and does not address the spectral limitations of SSTV. Now a natural question arises: \textit{can we enhance the noise removal capabilities of SSTV while preserving complex spatial structures and spectral jumps?}
%To address challenge 1) of SSTV, Graph Spatio-Spectral Total Variation (GSSTV)~\cite{Takemoto2022GSSTV} incorporates a spatial graph-based weighted difference operator into SSTV.
%By utilizing the spatial graph, GSSTV successfully captures more complex spatial structures compared to SSTV.
%However, GSSTV leaves noise in low-weight regions of the spatial graph , and does not resolve the spectral limitations of SSTV.
%Therefore, we pose the research question: can we enhance the noise removal capabilities of SSTV while preserving complex spatial structures and spectral jumps?

In this paper, we propose a novel method that incorporates both spatial and spectral graph-based TVs into SSTV, called Graph-Aided Spatio-Spectral Total Variation (GASSTV).
The main contributions of this article are listed below:
\begin{itemize}
    \setlength{\leftskip}{-10pt}
%    \item We design a novel regularization method, namely GASSTV. GASSTV integrates two weighted TVs based on spatial and spectral graphs separately from SSTV, thereby enhancing the noise removal capabilities of SSTV while capturing complex spatial structures and spectral jumps. Furthermore, we propose a robust approach to construct the spectral graph from the observed noisy HS image (Refer to GSSTV for constructing the spatial graph).
    \item We propose a novel regularization method, Graph-Aided Spatio-Spectral Total Variation (GASSTV), which incorporates spatial and spectral graph-based weighted TVs separately from SSTV. These graph-based TVs enhance the noise removal capabilities of SSTV while preserving complex spatial structures and spectral jumps. Furthermore, we robustly construct the spectral graph from the noisy HS image (Refer to GSSTV for constructing the spatial graph).
    \item We formulate the HS image denoising problem as a constrained convex optimization problem involving GASSTV. To simplify parameter tuning, we impose data fidelity and sparse noise characterization terms as constraints rather than adding them to the objective function.
    \item We develop an efficient algorithm based on Preconditioned Primal-Dual Splitting Method (P-PDS)~\cite{Pock2011PPDS, Naganuma2023PPDS} to solve our optimization problem. Unlike standard PDS~\cite{Chambolle2011PDS, Condat2013PDS}, P-PDS automatically determines appropriate stepsizes.
\end{itemize}
Finally, we demonstrate the effectiveness of GASSTV by comparing it with state-of-the-art HS image regularization methods through HS image denoising experiments.