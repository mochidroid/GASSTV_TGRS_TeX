\section{Introduction}
\IEEEPARstart{H}{yperspectral} (HS) images are three-dimensional data cubes, where each pixel contains a detailed spectrum spanning hundreds of contiguous wavelength bands from the ultraviolet to the near-infrared range. This rich spectral information can visualize materials and phenomena that are invisible in conventional RGB imagery. Furthermore, advanced analytical techniques, such as anomaly detection~\cite{Matteoli2014Anomaly, Su2022Anomaly}, classification~\cite{Ghamisi2017Classification,Li2019Classification,Nicolas2019Classification}, and unmixing~\cite{Bioucas-Dias2012UnmixingOverview,Ma2014UnmixingOverview}, enable the extraction of advanced knowledge from HS images, which can be applied in various fields including agriculture, mineralogy, astronomy, and biotechnology~\cite{Borengasser2007HSIApplications,Grahn2007Techniques, Thenkabail2016VegetationOverview,Lu2020AgricultureOverview}. Nevertheless, HS images are unavoidably degraded by various types of noise during acquisition arising from sensor defects, photon effects, atmospheric absorption, and other environmental factors~\cite{Shen2015DenoisingOverview, Rasti2018DenoisingOverview, Shen2022DenoisingOverview}. Such degradations can severely hinder the performance of subsequent analyses, making effective HS image denoising an indispensable preprocessing step for ensuring reliable outcomes.


Over the past decade, HS image denoising methods have evolved along two main lines: deep neural network (DNN)-based approaches and model-based regularization approaches. DNN-based methods, including convolutional neural networks (CNNs), recurrent models, and transformer-based architectures, have achieved remarkable progress by learning complex spatial–spectral dependencies directly from data.  Despite these capabilities, DNNs generally lack strong theoretical underpinnings, making their outputs difficult to interpret and their behavior hard to explain—an important drawback in applications where transparency and reliability are required. In contrast, model-based approaches explicitly encode the physical and statistical properties of HS images into mathematically well-defined formulations. While such methods require careful modeling of the complex nature of HS images, they can provide interpretable and theoretically grounded results. Motivated by these characteristics, this work focuses on advancing the model-based paradigm for HS image denoising.


Model-based approaches for HS image denoising can be broadly categorized into low-rank (LR) and total variation (TV) regularization frameworks. LR-based methods exploit the strong spectral correlations across all bands, modeling HS images as lying in a low-dimensional subspace. This enables the capture of nonlocal spectral relationships and effective suppression of random noise; however, LR models often incur high computational cost and may fail to preserve fine spatial details, especially when strong spatial structures are present. In contrast, TV-based methods are computationally efficient and have shown strong performance by modeling the spatial and spectral piecewise smoothness of HS images. Nevertheless, conventional TV formulations evaluate differences uniformly across adjacent pixels or bands, which limits their ability to preserve complex spatial structures and to capture abrupt spectral variations (spectral jumps), often resulting in the loss of important details. These limitations raise a natural research question: Can we design a TV-based regularization method that retains the efficiency and edge-preserving capabilities of TV while effectively capturing complex spatial structures and spectral jumps for robust HS image denoising? To answer this question, we would first like to provide a comprehensive review of existing TV-type regularization methods for HS images.


\subsection{Existing TV-type Regularization Methods}
\label{sec:existing_tv_methods}
% TVは自然画像のエッジを保存のために提案された
% HTVは自然画像の手法を拡張して~


%As a promising extension of SSTV, Graph Spatio-Spectral Total Variation (GSSTV)~\cite{Takemoto2022GSSTV} integrates a spatial graph-based weighted difference operator into SSTV, addressing challenge 1) i.e., preserving complex spatial structures. However, GSSTV leaves residual noise in low-weight regions of the spatial graph, and does not address the spectral limitations of SSTV. Now a natural question arises: \textit{can we design a regularization method that enhances the noise removal capabilities of SSTV while capturing complex spatial structures and spectral jumps?} %However, GSSTV leaves residual noise in low-weight regions of the spatial graph, and does not address the spectral limitations of SSTV. Now a natural question arises: \textit{can we enhance the noise removal capabilities of SSTV while preserving complex spatial structures and spectral jumps?}
%To address challenge 1) of SSTV, Graph Spatio-Spectral Total Variation (GSSTV)~\cite{Takemoto2022GSSTV} incorporates a spatial graph-based weighted difference operator into SSTV.
%By utilizing the spatial graph, GSSTV successfully captures more complex spatial structures compared to SSTV.
%However, GSSTV leaves noise in low-weight regions of the spatial graph , and does not resolve the spectral limitations of SSTV.
%Therefore, we pose the research question: can we enhance the noise removal capabilities of SSTV while preserving complex spatial structures and spectral jumps?

\subsection{Contribution}
\label{sec:contribution}
In this paper, we propose a novel method that incorporates both spatial and spectral graph-based TVs into SSTV, called Graph-Aided Spatio-Spectral Total Variation (GASSTV).
The main contributions of this article are listed below:
\begin{itemize}
    \setlength{\leftskip}{-10pt}
%    \item We design a novel regularization method, namely GASSTV. GASSTV integrates two weighted TVs based on spatial and spectral graphs separately from SSTV, thereby enhancing the noise removal capabilities of SSTV while capturing complex spatial structures and spectral jumps. Furthermore, we propose a robust approach to construct the spectral graph from the observed noisy HS image (Refer to GSSTV for constructing the spatial graph).
    \item We propose a novel regularization method, Graph-Aided Spatio-Spectral Total Variation (GASSTV), which incorporates spatial and spectral graph-based weighted TVs separately from SSTV. These graph-based TVs enhance the noise removal capabilities of SSTV while preserving complex spatial structures and spectral jumps. Furthermore, we robustly construct the spectral graph from the noisy HS image (Refer to GSSTV for constructing the spatial graph).
    \item We formulate the HS image denoising problem as a constrained convex optimization problem involving GASSTV. To simplify parameter tuning, we impose data fidelity and sparse noise characterization terms as constraints rather than adding them to the objective function.
    \item We develop an efficient algorithm based on Preconditioned Primal-Dual Splitting Method (P-PDS)~\cite{Pock2011PPDS, Naganuma2023PPDS} to solve our optimization problem. Unlike standard PDS~\cite{Chambolle2011PDS, Condat2013PDS}, P-PDS automatically determines appropriate stepsizes.
\end{itemize}
Finally, we demonstrate the effectiveness of GASSTV by comparing it with state-of-the-art HS image regularization methods through HS image denoising experiments.