\begin{abstract}
This paper proposes a novel regularization method, named \textit{Spatio-Spectral Structure Tensor Total Variation} ($\SSSTTV$), for denoising and destriping of hyperspectral (HS) images. HS images are inevitably contaminated by various types of noise, during acquisition process, due to the measurement equipment and the environment. For HS image denoising and destriping tasks, Spatio-Spectral Total Variation (SSTV) is widely known as a powerful regularization approach that models the spatio-spectral piecewise smoothness. However, since SSTV refers only to the local differences of pixels/bands, edges and textures that extend beyond adjacent pixels are not preserved during denoising process. To address this problem, we newly introduce $\SSSTTV$, which is designed to preserve two essential physical characteristics of HS images: semi-local spatial structures and spectral correlation across all bands. Specifically, we define $\SSSTTV$ as the sum of the nuclear norms of spatio-spectral structure tensors, which are matrices formed by arranging second-order spatio-spectral difference vectors within semi-local areas. Furthermore, we formulate the HS image denoising and destriping problem as a constrained convex optimization problem involving $\SSSTTV$ and develop an algorithm based on a preconditioned primal-dual splitting method to solve this problem efficiently. Finally, we demonstrate the effectiveness of $\SSSTTV$ by comparing it with existing methods, including state-of-the-art ones through denoising and destriping experiments. The source code is available at \url{https://github.com/MDI-TokyoTech/Spatio-Spectral-Structure-Tensor-Total-Variation}.


% This paper proposes a novel regularization method, named \textit{Spatio-Spectral Structure Tensor Total Variation} ($\SSSTTV$), for denoising and destriping of hyperspectral (HS) images. 	HS images are inevitably contaminated by various types of noise, during acquisition process, due to the measurement equipment and the environment. For HS image denoising and destriping tasks, Spatio-Spectral Total Variation (SSTV), defined by the $\ell_{1}$-norm of second-order spatio-spectral differences, is widely known as a powerful regularization approach that models the underlying spatio-spectral properties. However, since SSTV refers only to adjacent pixels/bands, semi-local spatial structures are not preserved during denoising process. To address this problem, we newly introduce $\SSSTTV$, defined using the nuclear norms of matrices formed by arranging second-order spatio-spectral difference vectors for each band in semi-local area (we call these matrices as spatio-spectral structure tensors). The design of this regularization function preserves the semi-local spatial structures and the spectral correlation across all bands while ensuring robust mixed noise removal. Furthermore, we formulate the HS image denoising and destriping problem as a convex optimization problem involving $\SSSTTV$ and develop an algorithm based on a preconditioned primal-dual splitting method to solve this problem efficiently. Finally, we demonstrate the effectiveness of $\SSSTTV$ by comparing it with existing methods, including state-of-the-art ones through denoising and destriping experiments.


%この正則化関数の設計は、半局所領域の空間構造とスペクトル相関を捉えながら
%The design of this regularization function can 
%This regularization function can simultaneously capture the spatial piecewise-smoothness, the spatial similarity between adjacent bands, and the spectral correlation across all bands in small spectral blocks.
%By doing so, our method preserves the semi-local spatial structures while ensuring robust noise removal.

%The design of the proposed regularization function enables the preservation of the semi-local spatial structures while effectively removing noise.
%Furthermore, our method captures the spectral correlation across all bands 
% The proposed regularization method simultaneously models the spatial piecewise-smoothness, the spatial similarity between adjacent bands, and the spectral correlation across all bands in small spectral blocks, leading to effective noise removal while preserving the semi-local spatial structures.

% Conventional
% This paper proposes a novel regularization method, named \textit{Spatio-Spectral Structure Tensor Total Variation} ($\SSSTTV$), for denoising and destriping of hyperspectral (HS) images.
% HS images are inevitably contaminated by various types of noise, during acquisition process, due to the measurement equipment and the environment.
% For HS image denoising and destriping tasks, Spatio-Spectral Total Variation (SSTV), defined using second-order spatio-spectral differences, is widely known as a powerful regularization approach that models the underlying spatio-spectral properties.
% However, since SSTV refers only to adjacent pixels/bands, semi-local spatial structures are not preserved during denoising process.
% To address this problem, we newly design $\SSSTTV$, defined by the sum of the nuclear norms of matrices consisting of second-order spatio-spectral differences in small spectral blocks (we call these matrices as spatio-spectral structure tensors).
% The proposed regularization method simultaneously models the spatial piecewise-smoothness, the spatial similarity between adjacent bands, and the spectral correlation across all bands in small spectral blocks, leading to effective noise removal while preserving the semi-local spatial structures.
% Furthermore, we formulate the HS image denoising and destriping problem as a convex optimization problem involving $\SSSTTV$ and develop an algorithm based on a preconditioned primal-dual splitting method to solve this problem efficiently. 
% Finally, we demonstrate the effectiveness of $\SSSTTV$ by comparing it with existing methods, including state-of-the-art ones through denoising and destriping experiments.
\end{abstract}